\documentclass[10pt,a4paper]{moderncv}

\usepackage{fullpage}
\moderncvstyle{classic}
\moderncvcolor{green}
\usepackage[margin=0.85in]{geometry}
%\usepackage[scale=0.75]{geometry}
\recomputelengths                             % required when changes are made to page layout lengths

\firstname{James}
\familyname{Howe}
\title{Curriculum Vitae}                           % optional, remove / comment the line if not wanted
\address{Department of Computer Science,}{University of Bristol,}{Bristol, BT8 1UB, UK}% optional, remove / comment the line if not wanted; the "postcode city" and and "country" arguments can be omitted or provided empty
%\mobile{+447817479003}                          % optional
%\phone{+2~(345)~678~901}                           % optional
%\fax{+3~(456)~789~012}                             % optional
\email{james.howe@bristol.ac.uk}                           % optional
\homepage{www.jameshowe.eu}                         % optional
%\extrainfo{additional information}                 % optional

\photo[64pt][0.4pt]{fullsizeoutput_1fe1.jpeg}              % optional

%\quote{Some quote}                                 % optional

\usepackage{lipsum}

\begin{document}
\vspace*{-2\baselineskip}
\makecvtitle
\vspace{-1.5\baselineskip}

\section{\textbf{Education}}
\cventry{2013--}{\textbf{PhD Computer Science}}{Queen's University}{Belfast}{}{Currently researching post-quantum cryptography in collaboration with Thales UK and SAFEcrypto. The research mainly focuses on secure and practical lattice-based cryptography, implementing in hardware and software. Supervised by Professor M\'{a}ire O'Neill.}  % arguments 3 to 6 can be left empty
\cventry{2011--2012}{\textbf{MSc Mathematics of Cryptography \& Communications}}{{\newline}Royal Holloway, University of London}{Surrey}{}{Thesis titled `The Cryptanalysis of Block Ciphers' obtained distinction, supervised by Professor Sean Murphy. Modules focused on pure mathematics, cryptography, and coding theory.}  % arguments 3 to 6 can be left empty
\cventry{2008--2011}{\textbf{BSc (Hons) Mathematics}}{University of Greenwich}{London}{}{Obtaining a first class degree. Thesis concentrated on the theory and practicality of modern cryptography. Modules focused on applied mathematics, probability, and statistics.}  % arguments 3 to 6 can be left empty

\section{\textbf{Work Experience}}
\cventry{2017--}{\textbf{Researcher}}{University of Bristol}{Bristol}{}{Supervised by Elisabeth Oswald, researching software and hardware optimised designs, as well as side-channel analysis, of post-quantum cryptography. Includes some teaching responsibilities.}  % arguments 3 to 6 can be left empty
\cventry{2013--2017}{\textbf{Teaching Assistant}}{Queen's University}{Belfast}{}{Demonstrating in mathematics and cryptography lectures at Queen's University.}  % arguments 3 to 6 can be left empty
\cventry{2009--2011}{\textbf{University Ambassador}}{University of Greenwich}{London}{}{Assisted college-level mathematics classes in schools across Greater London in partnership with the university. Also worked as an employee of the university; giving talks to prospective students in schools, giving tours of the campus, and participating at open days.}  % arguments 3 to 6 can be left empty
\cventry{2010--2011}{\textbf{Statistical Analyst}}{University of Greenwich}{London}{}{Worked with a number of major databases in the university's statistics department. Produced reports and presentations to use in meetings with the Chancellor and Vice-Chancellor. Updated statistics published on the university's website.}  % arguments 3 to 6 can be left empty

\section{\textbf{Achievements}}
\cventry{}{Publications}{}{}{}{(1) Howe, James, et al., ``Practical Lattice-based Digital Signature Schemes.'' \emph{ACM Transactions on Embedded Computing Systems (TECS)} 14.3 (2015): 41.{\newline}
(2) Howe, James, et al., ``Lattice-based Encryption Over Standard Lattices in Hardware.'' \emph{Design Automation Conference (DAC)}, 2016.{\newline}
(3) O'Neill, M\'{a}ire, et al., ``Secure Architectures of Future Emerging Cryptography SAFEcrypto.'' \emph{ACM International Conference on Computing Frontiers}, 2016.{\newline}
(4) Khalid, Ayesha et al., ``Time-Independent Discrete Gaussian Sampling For Post-Quantum Cryptography.''
\emph{Field-Programmable Technology (FPT)}, 2016.{\newline}
(5) Howe, James, et al., ``On Practical Discrete Gaussian Samplers For Lattice-Based Cryptography.''
\emph{IEEE Transactions on Computers}, vol. 67, no. 3, pp. 322-334, 2018.{\newline}
(6) Howe, James, et al., ``Compact and Provably Secure Lattice-Based Signatures in Hardware.'' \emph{IEEE International Symposium on Circuits and Systems (ISCAS)}, 2017, pp. 1-4.{\newline}
(7) Howe, James and O'Neill, M\'{a}ire ``GLITCH: A Discrete Gaussian Testing Suite For Lattice-Based
Cryptography'' \emph{SECRYPT}, 2017.{\newline}
(8) A. Khalid, et al., ``Compact, Scalable, and Efficient Discrete Gaussian Samplers for Lattice-Based Cryptography,'' \emph{IEEE ISCAS}, 2018, pp. 1-5.{\newline}
(9) Howe, James, et al., ``Standard Lattice-Based Key Encapsulation on Embedded Devices.'' \emph{IACR Transactions on Cryptographic Hardware and Embedded Systems}, 2018(3), 372-393.{\newline}
(10) Ravi, Prasanna, et al., ``Side-channel Assisted Existential Forgery Attack on Dilithium - A NIST PQC candidate.'' \emph{IACR Cryptology ePrint Archive}, Report 2018/821, 2018.{\newline}
(11+) More in submission.{\newline}
{\newline}
{\newline}
}

\cventry{}{Paper Reviews}{}{}{}{Reviewed papers for ACM TECS, IEEE Transactions on Computers, ASIACRYPT 2018, Designs Codes and Cryptography, CT-RSA 2018, CARDIS 2018, SAC 2016, WAHC 2015, and Security and Communication Networks.}
\cventry{}{Conferences}{}{}{}{Presented at conferences T-CHES 2018, ISCAS 2018, ISCAS 2017, DAC 2016, FPT 2016, and Tomorrow's Mathematicians Today 2009.} %Presented ``Standard Lattice-Based Key Encapsulation on Embedded Devices'' at the Lattice Coding & Crypto Meeting on 24 Sept 2018
\cventry{}{Invited talks}{}{}{}{Presented ``Standard Lattice-Based Key Encapsulation on Embedded Devices'' at the Lattice Coding \& Crypto Meeting on 24 Sept 2018.}
\cventry{}{Awards}{}{}{}{Received COST Action IC1306 stipend (2014) \& COST Action IC1306 STSM grant (2015).}
\cventry{}{IT Skills}{}{}{}{Proficient in all major operating systems (OSX/Linux/Windows) as well as mathematical tools such as
Mathematica, Matlab, and Minitab. Very competent with VHDL (ISE \& Vivado), Python, C/C++, with some Java and HTML (see \url{www.pqczoo.com}) experience.}  % arguments 3 to 6 can be left empty
%\cventry{}{Accolades}{}{}{}{University course representative at Royal Holloway and the University of Greenwich. Also treasurer and magazine editor of the mathematics society at the University of Greenwich. Awarded one of the top student workers as a university ambassador.}  % arguments 3 to 6 can be left empty
\cventry{}{Volunteering}{}{}{}{Working with the World Book Night charity since 2011. Tour guide at Bletchley Park, a museum for the national codes centre during WW2, in 2009. Mentor to first year undergraduates whilst at the University of Greenwich.}
%\cventry{}{Languages}{}{}{}{Native English speaker. Basic/Intermediate proficiency in German and basic knowledge of French.}
%\cventry{}{Hobbies}{}{}{}{Tennis, guitar, reading and cycling.}  % arguments 3 to 6 can be left empty

\vspace{0.25cm}

\section{\textbf{Research Statement}}

My research aims to bring principles and techniques from lattice-based cryptography to the design and implementation
of secure and correct systems. To this end, I became an expert in the theory of lattice-based cryptography, and then
designed architectures using optimisations which target specific devices such as FPGAs. In the past, I have studied
applied mathematics during my bachelor's degree and pure mathematics during my master's degree, so I am very
familiar with the mathematical side of cryptography. I am now looking to use my expertise to make cryptography
(ideally, lattice-based cryptography) more practical, to be used in real-world applications, for the benefit of secure
communications.\vspace{0.5cm}


I completed a research visit in 2015 at Ruhr-Universit\"{a}t Bochum with Thomas P\"{o}ppelmann and Tim G\"{u}neysu, funded by COST. For
this research we investigated new techniques for the discrete Gaussian sampling component, as well as techniques to
check the samplers validity and correct functionality. This collaboration also resulted in a joint journal publication on
lattice-based digital signature schemes.\vspace{0.5cm}

I also completed an internship with Thales Research and Technology in 2015 with Adrian Waller. During this
internship, I targeted a highly secure lattice-based cryptoscheme: standard LWE encryption. It was previously believed
that this scheme, in fact any standard lattice-based scheme, would perform badly in hardware. The hardware designs
I proposed in fact competed with the corresponding encryption scheme over ideal lattices, with a slight increase in
hardware resource consumption. This has been further improved after publication for inclusion in my PhD thesis, in
which the design consumes less area.\vspace{0.5cm}

As a PhD student and Research Assistant, I also work with SAFEcrypto (\url{http://www.safecrypto.eu}), which has ties
with a number of European research centres. The main outputs of this collaboration was a comprehensive evaluation
of discrete Gaussian samplers in hardware and a low-area hardware design of Ring-TESLA, an ideal lattice-based
signature scheme. The discrete Gaussian samplers proposed outperformed all previous work in hardware, as well as offering
constant run-time which is preferable due to side-channel analysis. The hardware designs of Ring-TESLA provide
generic hardware architectures, which allows ease of use with a number of different parameter sets. This hardware
architecture has the potential for high throughput with a fast NTT multiplier.\vspace{0.5cm}

I then joined the side-channel and cryptography group at the University of Bristol, supervised by Elisabeth Oswald. Here I learnt a lot about side-channel analysis, a new field I am very interested in researching, particularly with respect to lattice-based cryptography. During this time I also researched hardware and software designs for a potential NIST key encapsulation post-quantum standard, which I presented at T-CHES.\vspace{0.5cm}

For references, please contact: Prof. Maire O'Neill (\url{m.oneill@ecit.qub.ac.uk}) or Dr. Francesco Regazzoni (\url{regazzoni@alari.ch}).

\end{document} 